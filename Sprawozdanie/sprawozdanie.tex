% KLASA DOKUMANTU
\documentclass[11pt, a4paper]{article}

% PREAMBUŁA
\usepackage[]{amsmath,amssymb}
\usepackage[T1]{fontenc}
\usepackage[utf8]{inputenc}
\usepackage[]{graphicx}
\usepackage[polish]{babel}
\usepackage[]{lmodern}
\usepackage{hyperref}








	
\begin{document}
	
	\begin{center}
		\huge\textbf{Sprawozdanie z Projektu 1\\
			Transformacje współrzędnych}
	\end{center}
	\begin{center}
		\large Informatyka geodezyjna II \\
		13.05.2024
	\end{center}
	\vspace{\baselineskip}
	\vspace{\baselineskip}
	\vspace{\baselineskip}
	\vspace{\baselineskip}
	\noindent
	\Large Autorzy: Bartłomiej Cabaj, Igor Dudek\\
	Zajęcia: poniedziałek 12:15 - 14:00\\
	Nr grupy: 1\\
	
	\newpage
	\section{Cel ćwiczenia}
	Celem ćwiczenia było stworzenie skryptu implementującego transformacje współrzędnych między różnymi układami na kilku dostępnych elipsoidach.
	
	\section{Wykorzystane narzędzia i materiały}
	Do realizacji projektu wykorzystano:

	
	\begin{itemize}
		\item Python w wersji 3.11.5
		\item Git
		\item Środowisko programistyczne Spyder
		\item System operacyjny Windows 10/11
		\item Biblioteka numpy
		\item BIblioteka math
		\item Biblioteka sys
		\item Program Latex 
	\end{itemize}
	
	\section{Przebieg ćwiczenia}
	W trakcie ćwiczenia zaimplementowano algorytmy transformacji współrzędnych, korzystając z wzorów znalezionych w literaturze i materiałów udosępnionych na zajęciach. Utworzyliśmy plik w programie Spyder, storzyliśmy klase, która zawierała elipsoidy: wgs84,grs80,krasowski. Następnie napisaliśmy funkcjie wraz z ich dokumentacją:
	\begin{itemize}
		\item xyz ---> plh
		\item plh ---> xyz
		\item xyz ---> neu
		\item bl ---> uklad 2000
		\item bl ---> uklad 1992
		
	\end{itemize} 
	 Wprowadziliśmy klauzulę if \_\_name\_\_="\_\_main\_\_".
	 Użyliśmy biblioteki sys aby program dało się wywoływać z konsoli. Wprowadziliśmy możliwość wyboru: transformacji, pliku wejściowego, elipsoidy.
	 Następnie napisaliśmy kod, który odczytuje dane z pliku wejściowego dla każdej z transformacji i zapisuje je do pliku wyjściowego. Na koniec dopisaliśmy kod, który obsługuje przypadki gdy urzytkownik wprowadzi niepoprawne wartości.
	 
	 Wyniki z każdej transformacji sprawdzono za pomocą programu z innych zajęć, co potwierdzało ich poprawność.
	
	\section{Link do repozytorium}
	Rezultat pracy można znaleźć w repozytorium GitHub pod adresem:
	
	 \url{https://github.com/Dzik37/Projekt_bartek_igor.git}
	
	\section{Podsumowanie}
	Nabyte umiejętności:
	\begin{itemize}
		\item Pisanie kodu obiektowego w Pythonie
		\item Korzystanie z konsoli 
		\item Obsługa programu w Pythonie z poziomu konsoli
		\item Tworzenie dokumentów w LaTeX
		\item Współpraca zespołowa z wykorzystaniem GitHub
		\item Pisanie dokumentacji
		\item Opisywanie działania programów 
		\item Systematycznej pracy przez okres trwania projektu
	\end{itemize}
	
	Spostrzeżenia i trudności:
	\begin{itemize}
		\item Transformacja bl ---> uklad 2000 i uklad 1992 dla elipsoidy krasowskiego nie działa
		\item Znajdywanie małych błędów
		
	\end{itemize}
	\newpage
	\section{Bibliografia}
	
	\begin{itemize}
		\item \url{https://en.wikipedia.org/wiki/World_Geodetic_System#WGS84}
		\item \url{http://uriasz.am.szczecin.pl/naw_bezp/elipsoida.html}
		\item \url{https://en.wikibooks.org/wiki/PROJ.4#Spheroid}
		\item \url{https://en.wikibooks.org/wiki/PROJ.4#Spheroid}
		\item \url{https://notatek.pl/transformacja-wspolrzednych-geocentrycznych-odbiornika-do-wspolrzednych-topocentrycznych}
		\item \url{http://www.geonet.net.pl/images/2002_12_uklady_wspolrz.pdf}
	\end{itemize}
	
	
\end{document}
